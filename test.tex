\documentclass[10pt]{article}
\usepackage{amsmath,amsfonts,amsthm,amssymb}
\usepackage{fancyhdr}
\usepackage{chngpage}
\usepackage{color}
\usepackage{graphicx}
\usepackage{boxedminipage}
\usepackage{pbox}
\usepackage{lscape}


\title{GosHawk - Smart App \\ Documatation}
\author{Itay Parnafes \& Tomer Shalmon}

\begin{document}
\maketitle
\date
\newpage
\tableofcontents
\newpage
\section{Introduction}
GosHawk is a smart compact home security system based on open-source development platforms. \\
GosHawk was built and designed to be as simple as possible so that the ordinary user can set it up and use it without any previous knowledge. \\
The GosHawk mobile app strives to be as user-friendly as possible in order to provide an easy convenient user experience. 
\\ \\
The current version of GosHawk is an alpha version, designated to provide a proof-of-concept for a free open-source modular smart home security system, initially started as a graduation project for computer-science bachelor degree.

\newpage
\section{System Architecture}
\subsection{Backend}
\subsubsection{Arduino}
Arduino is an open-source hardware, software and microcontroller based development board.
The specific type used in GosHawk is \emph{Arduino Uno}.
The Arduino is fed from different sensors scattered around the house, analyses its inputs and synchronizes with the main server.
The arduino uses an ethernet shield\footnote{Another board assembled on top of the main board in order to provide ethernet based communication.} in order to communicate with the main server using \emph{UDP}\footnote{User Datagram Protocol - Transport layer contectionless protocol used to deliver messages over IP network.} over port 9898\footnote{By default. Can be configured differently.}.
\\ \\
\quad The Arduino code is written in \emph{C++} and is well commented in order to provide highly detailed code for other coders to integrate their own code.
\\ \\Current supported sensors are described in details in section 2.2.2.

\subsubsection{Sensors}
The system sensors were particulary chosen to give the maximum information on the house state in minimum cost. \\
The sensors has to be connected to the designated input ports in order to work. \\ \\
\begin{tabular}{| l | l | l | l |}
	\hline
	\textbf{Sensor} & \textbf{Analog/Digital} & \textbf{Port} & \textbf{Description} \\ \hline

\end{tabular}
\subsubsection{GoServer}
The server is the "heart" of the system and is in charge of communicating and synchronizing between all system components.
The server runs on a \emph{Raspberry Pi 2 Model B}\footnote{A credit card-sized single-board computer.} hardware running \emph{Ubuntu 16.04 Xential Server} operating-system. \\
The server runs several component, overwatched by a \emph{watchdog}\footnote{A process which runs using crontab and keeps all specified processes up and running.}. \\
\quad \subsubsection*{Synchronizer}
\quad A simple server which listens on UDP port 9898 and receives event messages from the \emph{Arduino} and updates the shared DB\footnote{Database}. \\
\quad The synchronizer was written in Python 2.7. \\
\quad \subsubsection*{Web Server}
\quad The web server is the main component of the GoServer and is designed to synchronize with the frontend components. \\
\quad The web server is designed to work solely with \emph{Android} clients\footnote{Currently iOS/web clients are not supported}, and provides fully integrated web app. \\
\quad The web server main jobs are:
\begin{itemize}
	\item Manage users registration and permissions
	\item Query shared DB as a backend component for the app
	\item Keep track of clients queries
	\item Check home presence of registered users
\end{itemize}
\quad The web server was written in \emph{Django}\footnote{Free and open-source web framework, written in Python, which follows the model–view–controller (MVC) architectural pattern.} 1.8.7 based on Python 2.7.

\quad \subsubsection*{Hawk-Eye}
\quad The Hawk-Eye is a python script used to take photos using the installed camera in the event of a breach. \\
\quad The Hawk-Eye was written in Python 2.7.


\subsection{Frontend}
\subsubsection{Android App}
blah blah blah....

\section{Requirements}
\begin{itemize}
	\item Arduino (Uno or other)
	\item Ethernet shield
	\item Keypad
	\item Small LCD screen
	\item Vibrations sensor
	\item Raspberry Pi
	\item IP Camera
\end{itemize}
\subsection{Raspberry Pi}
\begin{itemize}
	\item Ubuntu 16.04 Xenial Server
	\item Python 2.7	
	\item Django 1.8.7
	\item GoServer package
	\item Active internet connection
\end{itemize}

\section{System Components}
\subsection{Web Server}
\begin{landscape}
\subsubsection{Server Available Methods}
  \begin{adjustwidth}{-3cm}{} 
    \begin{tabular}{| l | l | l | p{3cm} | p{5.5cm} | l |}
      \hline
      \textbf{URI} & \textbf{Type} & \textbf{Parameters} & \textbf{Description} & \textbf{Response} & \textbf{Remarks} \\ \hline
      /app/login & POST & \pbox{30cm}{name \\ password} & A registered user login page & \pbox{25cm}{\{ \\
      'Access Granted': 'True',\\ 'uid': user\_id,\\ 'user\_type': permission \\ \}} & Cookies must be enables \\ \hline
     /app/register\_mac/[mac\_address] & POST & None & Register given mac to a registered user &\{'Success': 1\} & Request must be sent with auth cookie \\ \hline    
    /app/sync & GET & None & Request all unread events (by a specific user) & \pbox{25cm}{\{ \\ 
             	\quad events: [ \\ 
             		\qquad \{ \\ 
             			\qquad \quad timestamp: \{ \\ 
             				\qquad \quad \quad "hour": 7, \\ 
       					    \qquad \quad \quad "month": 3, \\ 
                            \qquad \quad \quad "second": 57, \\ 
             				\qquad \quad \quad "year": 2016, \\ 
             				\qquad \quad \quad "day": 5, \\ 
             				\qquad \quad \quad "minute": 48 \\ 
             			\qquad \quad \}, \\ 
             			\qquad \quad "type": "Breach", \\ 
             			\qquad \quad "description": "house has been breached", \\ 
             			\qquad \quad "severity": "Critical" \\ 
             		\qquad \quad \} \\ 
             	\qquad ] \\ 
             \quad \} \\
           } & \\ \hline
     /app/create\_new\_user] & POST & \pbox{25cm}{name \\ password \\ permission} & Add new user to the local system & \{'Success': 1\} & Only 'Administrator' can add new users \\ \hline

\end{tabular}
  \end{adjustwidth}
\end{landscape}
\end{document}
